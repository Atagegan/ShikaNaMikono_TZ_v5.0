\section{2010 - CHEMISTRY 2A ALTERNATIVE A PRACTICAL} \index{Past Papers!Chemistry! 2010}
%2010 requires students to answer two (2) of the following questions, including number 1.

\begin{enumerate}

\item[1.] You are provided with the following:\\
Solution \textbf{D} containing 6.90 g of T$_2$CO$_3$ per 0.50 dm$^3$ of solution\\
Solution \textbf{N} containing 1.55 g of hydrochloric acid per 200 cm$^3$ of solution\\
Methyl orange indicator solution.\\
\vspace{-6pt}

\textbf{Procedure:}\\
Put solution N in the burette. Pipette 20 cm$^3$ or (25 cm$^3$) of D into a titration flask. Add a few drops of methyl orange indicator. Titrate solution N from the burette against solution D in the titration flask to the end point. Note the burette reading. Repeat the procedure to obtain three more values and record the results as shown in the following table.\\

\begin{enumerate}
\item[(a)] Table of results\\
\vspace{-10pt}
\begin{enumerate}
\item[(i)] Burette readings\\
\begin{center}
\begin{tabular}{|l|p{2cm}|p{2cm}|p{2cm}|p{2cm}|} \hline
\textbf{Titration}&\multicolumn{1}{|c|}{\textbf{Pilot}}&\multicolumn{1}{|c|}{\textbf{1}}&\multicolumn{1}{|c|}{\textbf{2}}&\multicolumn{1}{|c|}{\textbf{3}}\\ \hline
Final reading (cm$^3$)&&&&\\ \hline
Initial reading (cm$^3$)&&&&\\ \hline
Volume used (cm$^3$)&&&&\\ \hline
\end{tabular}\\
\end{center}
\vspace{4pt}
\item[(ii)] The volume of the pipette used was \_\_\_\_ cm$^3$.
\vspace{2pt}
\item[(iii)] The volume of the burette used was \_\_\_\_ cm$^3$.
\vspace{2pt}
\item[(iv)] \_\_\_\_ cm$^3$ of solution D required \_\_\_\_ cm$^3$ of solution N for complete reaction.\\
\vspace{-8pt}
\item[(v)] The colour change at the end point was from \_\_\_\_ to \_\_\_\_.\\
\end{enumerate}
\vspace{-10pt}
\item[(b)] Write a balanced equation for the above neutralization reaction.\\
\vspace{-8pt}
\item[(c)] Calculate the following:\\
\begin{enumerate}
\vspace{-10pt}
\item[(i)] molarity of acid solution N.
\item[(ii)] molarity of the base solution D.
\item[(iii)] molar weight of T$_2$CO$_3$.
\item[(iv)] atomic mass of element T.
\end{enumerate}
\vspace{2pt}
\item[(d)] Identify element T in T$_2$CO$_3$
\end{enumerate}

\raggedleft \textbf{(25 marks)}\newpage

\raggedright

\item[2.] Sample \textbf{B} is a simple salt containing \textbf{one} cation and \textbf{one} anion. Carry out the experiments described in the following table carefully and record all your observations and appropriate inferences. Identify the cation and anion present in sample \textbf{B}.\\

\begin{center}
\begin{tabular}{|l|p{8cm}|l|l|}
\hline
\multicolumn{2}{|c|}{\textbf{Experiment}}&\textbf{Observation}&\textbf{Inference}\\ \hline
(a)&Appearance of sample B.&&\\ \hline
(b)&Put a spatulaful of sample B in a test-tube. Add water until half test-tubeful. Stir and divide the solution into five portions in different test tubes and do the following:&&\\ \cline{2-4}
&\begin{enumerate}
\item[(i)] add fresh zinc metal granules to the first portion. Heat for a while. Decant the result. Pour the solid material onto a filter paper and observe. Let it dry, then observe again.
\end{enumerate}&&\\ \cline{2-4}
&\begin{enumerate}
\item[(ii)] add NaOH solution until excess to the second portion then heat and observe again.
\end{enumerate}&&\\ \cline{2-4}
&\begin{enumerate}
\item[(iii)] add ammonia solution dropwise to the third portion until excess.
\end{enumerate}&&\\ \cline{2-4}
&\begin{enumerate}
\item[(iv)] add AgNO$_3$ to the fourth portion followed by dil. HNO$_3$.
\end{enumerate}&&\\ \cline{2-4}
&\begin{enumerate}
\item[(v)] add AgNO$_3$ to the fifth portion followed by ammonia solution.
\end{enumerate}&&\\ \hline
\end{tabular}\\

\end{center}


\textbf{Conclusion}\\

\begin{enumerate}
\item[(a)] The cation present in sample B is \_\_\_\_ and the anion is \_\_\_\_.
\item[(b)] What has been happening in the experiments (b)(i) and (b)(ii)? Use reaction equations where possible.
\end{enumerate}

\raggedleft \textbf{(25 marks)}

\raggedright

\item[3.] Substance Z contains \textbf{one} basic radical and and \textbf{one} acidic radical. Using systematic qualitative analysis procedures carry out experiments on sample Z and make appropriate observations and inferences to identify the radicals.\\

\begin{center}
\begin{tabular}{|p{4cm}|p{4cm}|p{4cm}|}
\hline
\textbf{Experiment}&\textbf{Observation}&\textbf{Inference}\\ \hline
&&\\
&&\\
&&\\
&&\\
\hline
\end{tabular}\\
\end{center}

\textbf{Conclusion}\\
\vspace{6pt}
The Basic radical in sample Z is \_\_\_\_ and the acidic radical is \_\_\_\_.\\


\end{enumerate}

\raggedleft \textbf{(25 marks)}\\

\raggedright
