\section{2011 - PHYSICS 2A ACTUAL PRACTICAL A}

\begin{enumerate}
\item The aim of this experiment is to determine the mass of a given dry cell size ``AA''. Proceed as follows:
\begin{itemize}
\item[(a)] Locate and note the centre of gravity $C$ of the metre rule by balancing it on the knife edge.
\item[(b)] Suspend the 50 g mass at length `$a$' cm on one side of the metre rule and the 20 g mass together with the dry cell at length `$b$' cm on the other side of the metre rule. Fix the 50 g mass at length 30 cm from the fulcrum and adjust the position of the 20 g mass together with the dry cell until the metre rule balances horizontally. Read and record the values of $a$ and $b$ as $a_0$ and $b_0$ respectively.
\item[(c)] Draw the diagram for this experiment.
\item[(d)] By fixing $a = 5$ cm from fulcrum $C$, find its corresponding length $b$.
\item[(e)] Repeat the procedure in (d) above for $a = 10$ cm, 15 cm, 20 cm and 25 cm. Tabulate your results.
\item[(f)] Draw a graph of `$a$' against `$b$' and calculate its slope $G$.
\item[(g)] Calculate $X$ from the equation $50 = \cfrac{b_0}{a_0}(20 + X)$.
\item[(h)] Comment on the value of $\cfrac{b_0}{a_0}$.
\item[(i)] Sate the principle governing this experiment.
\end{itemize}
\end{enumerate}

\flushright \textbf{(25 marks)}
\begin{enumerate}
\item[2.] You are provided with an ammeter, A, resistance box, R, dry cell, D, a key, K and connecting wires. Proceed as follows:
\begin{enumerate}
\item[(a)] Connect the circuit in series.
\item[(b)] Put $R$ = 1 $\Omega$ and quickly read the value of current $I$ on the ammeter.
\item[(c)] Repeat procedure (b) above for $R$ = 2 $\Omega$, 3 $\Omega$, 4 $\Omega$ and 5 $\Omega$. Record your results in a tabular form.
\item[(d)] Draw the circuit diagram for this experiment.
\item[(e)] Plot the graph of $R$ against $\cfrac{1}{I}$.
\item[(f)] Determine the slope of the graph.
\item[(g)] If the graph obeys the equation $R=\cfrac{E}{I}-r$, then
\begin{enumerate}
\item[(i)] suggest how $E$ and $r$ may be evaluated from your graph.
\item[(ii)] compute $E$.
\item[(iii)] compute $r$.
\end{enumerate}
\item[(h)] State one source of error and suggest one way of minimizing it.
\item[(i)] Suggest the aim of this experiment.
\end{enumerate}

\end{enumerate}

\flushright \textbf{(25 marks)}
\flushleft