\chapter{Preparation of Solutions} \index{Solutions! preparation of}
\label{cha:prep-solutions}

For many exercises, solutions do not need to be prepared accurately. Even a 50\% error in the preparation will still allow an effective experiment. For other activities, the solutions should be prepared with a great deal of accuracy. This is especially true for volumetric analysis and conductivity experiments. This section deals with the preparation of solutions when accuracy counts.

%==============================================================================
\section{Measure the Water}

\begin{itemize}

\item{Calculate the total volume of solution you need to prepare. For example, if you are doing a practical with 100 students and each requires 150~mL of solution you should make at least 15~L of solution. Making 20~L is probably wise, to have some extra.}

\item{Find a container large enough for the total volume. Plan ahead to ensure you have a large enough container.}

\item{Add the required volume of ordinary water.}

\item{If your syllabus encourages you to often practice acid-base titrations, designate a pair of suitably large buckets as your permanent ACID and BASE buckets and label them as such with a permanent pen. Then, use a 1 liter container to add water to these buckets, one liter at a time. Use the permanent pen to mark the water height after each liter. Use these marks when adding water to make solutions. Round up the volume you need to the nearest liter (e.g. 71 students $\times$ 200 mL per student = 14.2 L, so make 15 L). As long as you use relative standardization when you finish preparing the solutions, any errors you make when measuring the volume will not affect your students' results.}

\item{Distilled water is rarely necessary. If you are preparing solutions for volumetric analysis, read the section on Relative Standardization to learn how to correct small errors caused by the composition of the tap or river water. If the water forms a precipitate when making solutions of hydroxide or carbonate, allow the precipitate to settle and decant the solution. If you are making a dilute solution, you might add hydroxide or carbonate gradually with mixing until precipitation stops and then add the amount you actually need to the liquid after decantation. If the only water supply if muddy, let the dirt settle and decant or use a cloth filter. If the particles are very fine, add a chemical like potassium aluminum sulfate (alum) or iron sulfate to precipitate the dirt. If you think that you do need distilled water, rain water is almost always sufficient.}
\end{itemize}

What comes next depends on the nature of your stock chemical. In general, there are two kinds of solutions:
\begin{itemize}
\item{Solutions prepared from solid stock chemicals, e.g. sodium hydroxide, citric acid}
\item{Solutions prepared from liquid stock chemicals, e.g. sulfuric acid}
\end{itemize}

%==============================================================================
\section{Preparing solutions from solid stock chemicals}

\begin{itemize}

\item{Calculate the amount of solid chemical required. If the instructions give the required concentration in grams per liter (e.g. $ 4~^\text{g}/_\text{L} $ NaOH solution), multiple the total volume by the required concentration (e.g. $ 4~^\text{g}/_\text{L} \times 10~\text{L} = 40~\text{g} $). If the instructions give the required concentration in molarity or moles per liter (e.g. 0.1 M~NaOH solution), multiple the required molarity by the molecular mass of the compound to find the required concentration in grams per liter (e.g. $ 0.1~ ^\text{mol}/_\text{L} \times 40~^\text{g}/_\text{mol} = 4~^\text{g}/_\text{L} $). Then, multiple the required concentration by the total volume ($ 4~^\text{g}/_\text{L} \times 10~\text{L} = 40~\text{g} $).}

\item{Use a balance to weigh the solid chemical. Remember to weigh the chemical in a plastic container or on a sheet of paper and not on the scale pan directly. Some chemicals (e.g. sodium hydroxide) react with the metal pan. If you are unfamiliar with how to use a balance, read \nameref{cha:use-beam-balance} (p.~\pageref{cha:use-beam-balance}). If you do not have a balance, read the section on \nameref{cha:prep-solns-wo-bal} (p.~\pageref{cha:prep-solns-wo-bal}).}

\item{Carefully add the solid chemical to the water and stir with something unreactive (e.g. glass rod, broken burette, thick copper wire) until it has completely dissolved.}

\end{itemize}

%==============================================================================
\section{Preparing solutions from liquid stock solutions}

\begin{itemize}

\item{Calculate the amount of liquid chemical required. To do this, you need to know the molarity of your stock chemical. See the section on \nameref{cha:mol-bottle-liq} (p.~\pageref{cha:mol-bottle-liq}). If the instructions give the required concentration in molarity or moles per liter, use the dilution equation to calculate the amount of concentrated required:

\[ (M_{concentrated})(V_{concentrated}) = (M_{dilute})(V_{dilute}) \]

rearranging

\[ V_{dilute} = \frac{(M_{concentrated})(V_{concentrated})}{M_{dilute}} \]

For example, if you need 10~L of 0.1~M HCl and you have 12~M stock solution, the required volume of concentrated acid is

\[ V_{dilute} = \frac{(12~\text{M})(10~\text{L})}{0.1~\text{M}} \]

}

\item{If the instructions give the required concentration in grams per liter, divide this concentration by the molecular mass to get molarity (e.g. $ \frac {3.65~^\text{g}/_\text{L}}{36.5~^\text{g}/_\text{mol}} = 0.1~^\text{mol}/_\text{L}$) and then use the dilution equation as above.}

\item{Use a DRY measuring cylinder the measure the required amount of liquid chemical. Concentrated acids may be measured in standard lab grade plastic measuring cylinders – there is no need for glass. If you do not have a measuring cylinder, you can use a plastic syringe. Be sure to use the Air Cushion Method for measuring volumes with syringes (see the section on \nameref{cha:usesyringe}, p.~\pageref{cha:usesyringe}) – concentrated acids will rapidly corrode the rubber in the syringe on contact, causing the syringe to jam and become dangerous. Also, please read the description of \nameref{sub:conc-acids} in \nameref{cha:dangerchem} (p.~\pageref{sub:conc-acids}).}

\item{Carefully pour the liquid chemical into the container of water. Stir with something non reactive (glass rod, broken burette, thick copper wire) for about one minute.}

\end{itemize}

Then, for all volumetric analysis solutions, use the instructions in the \nameref{cha:rel-stan} (p.~\pageref{cha:rel-stan}) section to perfect the mole ratio of your solutions.