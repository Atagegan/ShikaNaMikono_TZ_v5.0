\chapter{Repairing Burettes}
\label{cha:burettes}

First, if you need burettes, consider buying plastic burettes. They are widely available if you ask persistently and they tend not to break. This may be hard as many suppliers prefer to sell glass burettes. Why? As one supplier told us, ''Because when people buy plastic burettes, they don’t return.''

The good news for every school with glass burettes is than often broken burettes can be repaired.

\section{The top of the burette is broken, 
above the 0~mL line.}

This burette is still fully functional. 
A student will probably need a beaker for filling the burette, 
but she should be using one anyway. 
Use a metal file (best!), 
stone, 
or piece of cement to gently grind the broken edge smooth to prevent cuts.

\section{The burette is broken in the graduated section, 
that is, 
between 0~ml and 50~ml.}
This burette is still slightly useful for titrations 
if it has most of its length. 
Students will just have an initial volume of 7~ml, 
perhaps. 
If it has broken around the 45~ml mark, 
no such luck. 
The burette tube however, 
is still quite useful as a glass pipe. 
Keep it around for other kinds of experiments. 
At the very least you have a glass rod for mixing solutions. 
Regardless, 
grind the edges smooth as in case one.

\section{The burette is broken below the 50~ml but above the valve.}
To fix this, 
you need a Biafa (fake Bic) pen and about 8cm of rubber tubing. 
Orange gas supply tubing is best, 
but hard to find. 
The black rubber of the inside of bicycle pump hoses also works. 
Large bike supply shops often have broken pumps 
with which they are willing to part for free. 
First, 
cut off the tip of the pen, 
the first 2~cm of so, 
and attach the non-tapered end it to the tubing. 
Cutting is easiest done by scoring all the way around 
with a razor blade and then cleanly snapping the shaft. 
Remove any plastic burrs from the cut edge 
and then insert the wider end of the severed tip 
into the plastic tubing so the narrow end hangs out. 
Second, 
remove from the pen the little plastic end cap 
(the one that tells you what color ink you have) 
and insert it into the tubing, 
curved side first. 
Push it about half way down the tube using your 
fingers like esophageal peristalsis and make sure that 
the axis of symmetry of the pen cap stays aligned 
with that of the rubber tubing. 
That is, 
if the now discarded pen were still there, 
it would be surrounded by the tube. 
Finally, 
attach the other end of the tubing to the broken burette. 
Again, 
grind the sharp glass end to smooth it. 
What you should end up with is a burette that does not pass solution 
except when you press on the tubing around the pen end cap, 
deforming the tube to allow liquid to pass. 
With practice this can be easier than using a valve, 
and just as accurate.

Steel ball bearings are available for cheap at bicycle supply shops. 
These might be an alternative to the end caps of Biafa pens 
if you can get them in the right size. 
Experiment!

\section{The valve is jammed}
No problem! Soak it in dilute acid (not nitric) until it is free.

\section{Case Five: The valve is hopelessly broken.}
Break the burette just above the valve and follow the instructions above. 
Soak a string in something flammable -- kerosene, 
nail polish remover -- and gently squeeze out the excess. 
Tie the string around the shaft where 
you want to ''cut'' the glass and remove the excess string. 
Dry up any liquid that spilled on other parts of the glass. 
Light the string on fire and rotate to make sure it burns evenly. 
After five or so seconds of burning, 
plunge the piece into a beaker or bucket of water. 
The contraction of the rapidly cooling glass 
should break the burette along where you tied the string. 
Grind the edge to smooth it.

\section{The burette is broken below the valve.}
This problem is mostly aesthetic, 
but to fix it you only need about 3~cm of rubber tubing 
and a clear plastic pen. 
Cut the tip from the pen as above and insert it into the tubing. 
Then stick the other end of the tubing onto the broken burette, 
grinding down the glass edge before you do.

\section{The rubber tubing is cracking.}
This usually comes from leaving clamps on the tubing during storage. 
To fix this, 
replace the rubber tubing. 
But while you are at it, 
insert a pen cap as in case three and do away with the clamps. 
They are more difficult to use and not as sensitive.
